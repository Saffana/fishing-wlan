\documentclass[a4paper, 11pt]{article}
%--Packages pour la mise en page--%
\usepackage[utf8x]{inputenc}
\usepackage[T1]{fontenc}
\usepackage[french]{babel} 
\usepackage{lmodern}
\usepackage{fullpage}
\usepackage[]{algorithm2e}
%---------------------------------%

%----Informations du document-----%
\title{Overview}
\author{Authors}
\date{04 Septembre 2015} 
%---------------------------------%

\begin{document}

\maketitle

\section*{Nom du projet :}
Récupération de données privées par hameçonnage.

\section*{Objectifs :} 

\begin{itemize}
	\item[•] L'objectif final de l'attaque sera la récupération des identifiants du compte Gmail de la cible. 
	\item[•] Pour cela, nous allons intercepter et détourner le trafic de la cible en utilisant les techniques du "man in the middle" (ARP Spoofing et DNS Spoofing).
	\item[•] La cible sera donc redirigée vers une page web pirate hébergée sur un serveur distant.
	\item[•] Pensant être sur une page de connexion Gmail, la victime va s'identifier et nous pourrons récupérer les données de la cible.
	\item[•] Expliquer les différentes contre-mesures pour ce genre d'attaques.
 	\item[•] [Bonus] Modifier les certificats et gérer HTTPS.
\end{itemize}

\section*{Ce qu'on va apprendre en mettant en place ce projet : }
\begin{itemize}
	\item[•] Mettre en place un "man in the middle" : maitrise des requêtes de niveau 2 (ARP)
	\item[•] Mettre en place un faux DNS : DNS Spoofing
	\item[•] Utilisation de Wireshark
	\item[•] Exploiter un serveur web avec une base de donnée
\end{itemize}

\section*{Partage du code : }
Le partage du code se fera via git.

\end{document}
